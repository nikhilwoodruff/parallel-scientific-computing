\documentclass[twocolumn]{article}
\usepackage{fancyhdr}
\renewcommand{\footrulewidth}{0.5pt}
\begin{document}
    \pagestyle{fancy}
    \fancyhead[L]{Parallel Scientific Computing}
    \fancyhead[R]{Lecture 1}
    \fancyfoot[L]{Computer Science}
    \fancyfoot[C]{}
    \fancyfoot[R]{Page \thepage}
    \setcounter{section}{0}
    \section{Introduction}
    Generally, when answering a question with a simulation process, we do the following:
    \begin{enumerate}
        \item Define the process we want to model.
        \item Build a mathematical model of the process.
        \item Write a numerical algorithm solving the model.
        \item Write simulation code implementing the algorithm.
        \item Visualise results.
        \item Produce a statement, or a tool from the results.
    \end{enumerate}
    Each step may involve going backwards, in order to validate previous steps.
    \subsection{Models}
    A model is a simplified representation of a real system. It is a mathematical description of the system, which is used to predict the behaviour of the system. For example, ordinary or partial differential equations are common models for natural processes:
    \[F = ma = m \frac{d^2x}{dt^2}\]
    A few examples of models:
    \begin{itemize}
        \item The Millenium-XXL Project is an $N$-body methods model simulating the generation of galaxy clusters in order to evaluate how plausible the "cold dark matter" hypothesis is.
        \item Another $N$-body methods model: Particulate Flow Simulation. This simulates blood flow in order to better understand issues caused by deformed blood cells.
        \item Micro and nano simulations are useful for modelling environments where usual patterns no longer hold.
    \end{itemize}
    Many models are based on differential equations. These usually have some core assumption of \emph{equilibrium} (where some derivative is equal to zero). For example, a heat transfer model might assume that in a model involving vertices connected by edges, the temperature of each vertex is equal to the average of the temperatures of its neighbours.
\end{document}